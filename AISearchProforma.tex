\documentclass[11pt,a4paper]{article}
\usepackage[margin=1in]{geometry}
\usepackage[T1]{fontenc}
\usepackage{lmodern}

\begin{document}
Name: Alexander McKinney \hspace{200pt} User-ID: vvvm23

Algorithm A: A* Search (with Greedy Search Heuristic)

Algorithm B: IDA* Search (with Greedy Search Heuristic)

\vspace{10pt}

Description of enhancement of Algorithm A: 2-opt tour optimisation

Description of enhancement of Algorithm B: 3-opt tour optimisation

\vspace{5pt}


\noindent\hrulefill

\vspace{5pt}

2-opt is a simple local search algorithm that attempts to solve the travelling salesman problem. I will use it instead to take our existing tour produced by either A-Star or IDA-Star Search and try to produce a better tour.

2-opt iteratively picks two cities $a$ and $b$ from tour $T$ and removes a sub tour consisting of cities from $a$ to $b$ in the tour inclusive. It will then reverse this subtour and insert it back into the tour $T$ to produce $T'$. If the tour length of $T'$ is less than $T$ then we replace $T$ with $T'$.

This repeats through all pairs of tours, and will continue looping until a pass is made where no better tours are found. Then, $T'$ is outputted.

2-opt on its own is a poor performing algorithm, but when we use a more sophisticated algorithm first, it can lead to small to medium decreases in tour length. Additionally, it works for any algorithm, as the only input is the tour we have already generated.

\vspace{5pt}

\noindent\hrulefill

\vspace{5pt}


3-opt is very similar to 2-opt in concept. It is also a local search algorithm for the travelling salesman problem. The difference is that instead of selecting two edges, 3-opt instead selects 3. (This can be generalised to $k$ edges for $k$-opt). It deletes these edges and then tries to reconnect the tour in all possible ways and selects the best. (as opposed to 2-opt which only has one such option)

This increases the time complexity from $O(n^2)$ to $O(n^3)$. My theory is that despite the increased worst case time complexity the actual time taken will still be acceptable as we will first use IDA* to produce a good tour before passing to 3-opt. This means that few iterations will happen before 3-opt naturally terminates leading to an acceptable running time. 

\vspace{5pt}
\noindent\hrulefill


\vspace{5pt}

\section*{References:}

\begin{itemize}
	\item G. A. CROES (1958). A method for solving travelling salesman problems. Operations Res. 6 (1958), pp., 791-812 
\item Christian Nilsson. Heuristics for the Traveling Salesman Problem.
\end{itemize}

\end{document}
